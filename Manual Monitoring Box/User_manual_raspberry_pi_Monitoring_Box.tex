\documentclass[11pt]{report}
\usepackage{graphicx}
\usepackage{pgfplots}
\usepackage{amsfonts}
\usepackage[all]{xy}
\usepackage{amsmath}
\usepackage{amssymb}
\usepackage{tikz}
\usepackage{forest}
\usepackage{tikz}
\usepackage{caption}
\usepackage{listings}
\usepackage{hyperref}
\usepackage{float}
% we want ER + above/below + left/right
\usetikzlibrary{er,positioning}
\usepackage{hyperref}
%Gummi|065|=)
\title{\textbf{MonitoringBox User manual}\\Raspberry pi}
\author{Team MonitoringBox}
\date{2017}
\begin{document}


\definecolor{mygreen}{rgb}{0,0.6,0}
\definecolor{mygray}{rgb}{0.5,0.5,0.5}
\definecolor{mymauve}{rgb}{0.58,0,0.82}
\definecolor{bg}{rgb}{0.9,0.9,0.9}

% https://en.wikibooks.org/wiki/LaTeX/Source_Code_Listings
\lstset{ %
  backgroundcolor=\color{bg},   % choose the background color
  basicstyle=\footnotesize,        % size of fonts used for the code
  breaklines=true,                 % automatic line breaking only at whitespace
  captionpos=b,                    % sets the caption-position to bottom
  commentstyle=\color{mygreen},    % comment style
  escapeinside={\%*}{*)},          % if you want to add LaTeX within your code
  keywordstyle=\color{blue},       % keyword style
  stringstyle=\color{mymauve},     % string literal style
  framesep=10pt,xleftmargin=10pt,xrightmargin=10pt
}

\maketitle

\tableofcontents{}

\chapter{Introduction}
	The manual describes how to use the MonitoringBox. For instructions of building the device see the setup manual.
\chapter{Device}
	\section{Parts}
		The MonitoringBox is made up from an Raspberry Pi and Arduinos connected to it. The Arduinos are in turn connected to the real sensors, these sensors
\chapter{Device interface}
\chapter{Web interface}
\end{document}
