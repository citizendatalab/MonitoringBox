\documentclass{report}
\usepackage[utf8]{inputenc}

\title{The Monitoring Box \\ 
Research on how to create an open source data-gathering platform for the Citizen Data Lab
\\ or  \\
Developing an open source, multi-research, geolocation aware, sensor data gathering platform usable by the students and researchers from the Makerslab on the Amsterdam University of Applied Sciences.}

\author{Mick Nieman \\ Amsterdam University of Applied Sciences \and
 Pjotr Scholtze \\ Amsterdam University of Applied Sciences \and
 Heeyeon Joung \\ Seoul National University of Science and Technology \\Amsterdam University of Applied Sciences}
 
\date{January 2017}

\begin{document}

\maketitle

\begin{abstract}
This has no content yet and needs to be filled in.     
\end{abstract}

\tableofcontents

\chapter{Introduction}
This report is written in commision of the Citizen Data lab from Amsterdam. The main goal of this report is to avoid pitfalls and find solutions for making a open source data-gathering platform. Commisioner of this research is Wouter Meys, Lab Coordinator of the Citizen Data Lab. The reason for this research, along with a developed prototype is the lack of useful environmental data-loggers which also keep track of the GPS-coordinates. This is critical for the Citizen Data Lab when they are, what they call, 'mapping the city'. 

\section{Statement of Purpose}
This has no content yet and needs to be filled in. 

\section{Statement of Significance}
This has no content yet and needs to be filled in. 

\chapter{Research Questions}

\section{Overall research question}
What factors contribute to creating an open source, multi-research, sensor geolocation aware, data gathering platform that can be used by the students and researchers from the Makerslab on the Amsterdam University of applied sciences?

\section{Sub-questions}
\begin{enumerate}
\item How technical are the students and researchers from the Makerslab on the Amsterdam University of applied sciences?
\begin{enumerate}
\item What documentation is needed and how detailed should it be
\end{enumerate}
\item What improves the usability of the product when taking the structure plane of user experience into account?
\item How can the data be made available such that the students \& researchers from the Minor Makerslab can use it?
\end{enumerate}

\chapter{Review of literature}
This has no content yet and needs to be filled in. 

\chapter{Methods}
This has no content yet and needs to be filled in. 

\chapter{Results}
\begin{enumerate}
\item How technical are the students and researchers from the Makerslab on the Amsterdam University of applied sciences?
\begin{enumerate}
\item What documentation is needed and how detailed should it be
\end{enumerate}
 We wrote 'Manual Monitoring Box' documentation based on the subjective answer to 'What documentation is needed and how detailed should it be'. We started with a list of parts we used and a brief introduction to them. Since we thought there might be some people who are new to the Arduino program, we explain how to install the software, how to upload it to Arduino, and how to get the results. And for each sensor, we describe how the Arduino and the sensor should be connected by schematic and letter. We invited two researchers and three students from the Amsterdam University of applied sciences and require them building the Monitoring box based on our 'Manual Monitoring Box'. \\
As a result, their technical level varied. Each student and researcher had their background knowledge, but they were all different. There is no mention of how to use the breadboard and how to soldering in the manual, but two researchers commented on it. One student pointed out the order of pictures and text. We added the picture of schematics and wrote the explain of schematics after the picture. However, the number in the picture was not clear, which caused a mistake. In fact, one researcher had difficulties in trying to connect only with the picture, and one student also had difficulty connecting, even though he had several experience of Arduino.\\
In conclusion, to create a monitoring box, you need a piece of hardware such as a circuit and a software knowledge such as raspberry pie and arduino. The range of hardware and software is very large and we can not measure and decide someone's technical level through the monitoring box buildability interview. However, depending on their own experience and background knowledge, there may be a first-hand knowledge of re-building this monitoring box. So the documentation for the students and researchers form the Makerslab on the Amsterdam University of applied sciences should be as detailed as possible including the structure of the plate, the principle of circuit connection, the necessary preparations and precautions for soldering.\\
 
\item What improves the usability of the product when taking the structure plane of user experience into account?

In interaction design in the structure plane of user experience, users must communicate correctly to the monitoring box and the monitoring box must deliver the information that the user wants immediately and accurately. In this respect, the Monitoring box has a touch screen and we have added four menus in this screen so that when the user has the information they want, they can instantly check the menu. The home screen shows what percentage of the storage capacity is in use and if users select 'Sensor' menu from the Monitoring Box, users can see the currently connected sensor. And if users select 'Wifi' menu from Monitoring Box, users can confirm the Wifi name and password.//
In information architecture in the structure plane of user experience, it should facilitate intuitive access to data. So we used wireless WiFi so that we could check the information if we had an internet capable device. And with a single push of a button on the Internet Web site, users are able to download the data of the sensors they had recorded at a glance. In addition, the monitoring box screen design allows intuitive use of the menu.


\item How can the data be made available such that the students \& researchers from the Minor Makerslab can use it?

We thought that in order for users to easily access data, they had to be able to see the data at a glance and approach it in an uncomplicated way. So we make the data users have recorded download at once and use the wireless data transmitter to view the data, even if users do not have a certain cable, users can see data only if they have a laptop or smart phone nearby. \\
There are some steps for Makerslab students and researchers to use. The monitoring box is equipped with Wi-Fi function. First, use a computer, laptop or mobile phone to connect to the monitoring box WiFi. And users can access the monitoring box web page by accessing 'monitoring.box:5000' on the web page. In this web page, users can see which sensor is connected now and download recoded data. 
\end{enumerate}
\chapter{Discussion}
This has no content yet and needs to be filled in. 

\chapter{Limitations}
This has no content yet and needs to be filled in. 

\chapter{Conclusion}
This has no content yet and needs to be filled in. 

\bibliography{Research_Report_Monitoring_Box}


\end{document}
