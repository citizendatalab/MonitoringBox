\documentclass[conference]{IEEEtran}
\usepackage[utf8]{inputenc}
\usepackage{blindtext, graphicx}


\hyphenation{op-tical net-works semi-conduc-tor}

\begin{document}

\title{The usability and buildability of an open source Monitoring Box}

\author{
	\IEEEauthorblockN{Mick Nieman}
	\IEEEauthorblockA{Business IT \& Management\\Amsterdam University\\of Applied Sciences\\
		Wibautstraat 2-4 1091GM Amsterdam\\
		The Netherlands\\
		Mick.Nieman@hva.nl
		}		
	\and
		\IEEEauthorblockN{Pjotr Scholtze}
		\IEEEauthorblockA{Software Engineering \\Amsterdam University\\of Applied Sciences\\
			Wibautstraat 2-4 1091GM Amsterdam\\
			The Netherlands\\
			Pjotr.Scholtze@hva.nl
			}			
		\and
			\IEEEauthorblockN{Heeyeon Joung}
			\IEEEauthorblockA{Electrical Engineering\\Seoul National University\\
			of Science and Technology \\
			Seoul Nowon-gu, Gongneung-dong,\\
			Gongneung-ro 232, South-Korea\\
			Julia.Joung@hva.nl
			}	
		}
	

\maketitle



\begin{abstract}
This paper is written in commision of the Citizen Data lab from Amsterdam. The main goal of this report is to avoid pitfalls and find solutions for making a open source data-gathering platform. Commisioner of this research is Wouter Meys, Lab Coordinator of the Citizen Data Lab. The reason for this research, along with a developed prototype is the lack of useful environmental data-loggers which also keep track of the GPS-coordinates. This is critical for the Citizen Data Lab when they are, what they call, 'mapping the city'. \\

\end{abstract}

\begin{IEEEkeywords}
Open-source, usability, buildability, environment, data logging
\end{IEEEkeywords}

\IEEEpeerreviewmaketitle

\section{Introduction}
 

\subsection{Statement of Purpose}
This has no content yet and needs to be filled in. 

\subsection{Statement of Significance}
This has no content yet and needs to be filled in. 

\section{Research Questions}

\subsection{Overall research question}
What factors contribute to creating an open source, multi-research, sensor geolocation aware, data gathering platform that can be used by the students and researchers from the Makerslab on the Amsterdam University of applied sciences?

\subsection{Sub-questions}
\begin{enumerate}
\item How technical are the students and researchers from the Makerslab on the Amsterdam University of applied sciences?
\begin{enumerate}
\item What documentation is needed and how detailed should it be
\end{enumerate}
\item What improves the usability of the product when taking the structure plane of user experience into account?
\item How can the data be made available such that the students \& researchers from the Minor Makerslab can use it?
\end{enumerate}

\section{Review of literature}
This has no content yet and needs to be filled in. 

\section{Methods}
This has no content yet and needs to be filled in. 

\section{Results}
This has no content yet and needs to be filled in. 

\section{Discussion}
This has no content yet and needs to be filled in. 

\section{Limitations}
This has no content yet and needs to be filled in. 

\section{Conclusion}
This has no content yet and needs to be filled in. 

\bibliography{Research_Report_Monitoring_Box}

\appendices


\end{document}
