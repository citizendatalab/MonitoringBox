\documentclass[conference]{IEEEtran}
\usepackage[utf8]{inputenc}
\usepackage{blindtext, graphicx}


\hyphenation{op-tical net-works semi-conduc-tor}

\begin{document}

\title{The usability and buildability of an open source Monitoring Box}

\author{
	\IEEEauthorblockN{Mick Nieman}
	\IEEEauthorblockA{Business IT \& Management\\Amsterdam University\\of Applied Sciences\\
		Wibautstraat 2-4 1091GM Amsterdam\\
		The Netherlands\\
		Mick.Nieman@hva.nl
		}		
	\and
		\IEEEauthorblockN{Pjotr Scholtze}
		\IEEEauthorblockA{Software Engineering \\Amsterdam University\\of Applied Sciences\\
			Wibautstraat 2-4 1091GM Amsterdam\\
			The Netherlands\\
			Pjotr.Scholtze@hva.nl
			}			
		\and
			\IEEEauthorblockN{Heeyeon Joung}
			\IEEEauthorblockA{Electrical Engineering\\Seoul National University\\
			of Science and Technology \\
			Seoul Nowon-gu, Gongneung-dong,\\
			Gongneung-ro 232, South-Korea\\
			Julia.Joung@hva.nl
			}	
		}
	

\maketitle	



\begin{abstract}
This paper is written in commission of the Citizen Data lab from Amsterdam. The main goal of this report is to avoid pitfalls and find solutions for making an open source data-gathering platform. Commissioner of this research is Wouter Meys, Lab Coordinator of the Citizen Data Lab. The reason for this research, along with a developed prototype is the lack of useful environmental data-loggers which also keep track of the GPS-coordinates. This is critical for the Citizen Data Lab when they are, what they call, 'mapping the city'. \\

\end{abstract}

\begin{IEEEkeywords}
Open-source, usability, buildability, environment, data logging
\end{IEEEkeywords}

\IEEEpeerreviewmaketitle

\section{Introduction}
 Researchers nowadays use data to answer the questions asked within their research, that is how research is done. Because of this collecting data for their research is also part of the researchers their task-list and that is not as easy as it seems. Data is collected by researchers doing so called 'data-sprints', this is where a group of people try to collect as much data as possible to do new findings on a certain topic. This is not different for the Citizen Data Lab and their researchers who do research, and back their research, by collecting data through data-sprints. Lately they find their selves struggling collecting the data and most important a specific aspect of their data, the location. For this very reason the development of the Monitoring Box began. The monitoring box tries to hand a solution to the researchers that not only is very convenient but also something researchers can build individually and without any interference of third parties. This way the costs stay low and the usability high. \par
This paper is a result of findings made during the development phase and the testing phase of the prototype of the monitoring box. The paper is written based on the findings and used to be learnt from when further developing the monitoring box. It may also be used by others developing similar open source data gathering platforms.

\subsection{Statement of Purpose}
This has no content yet and needs to be filled in. 

\subsection{Statement of Significance}
This has no content yet and needs to be filled in. 

\section{Research Questions}

\subsection{Overall research question}
What factors contribute to creating an open source, multi-research, sensor geolocation aware, data gathering platform that can be used by the students and researchers from the Makerslab on the Amsterdam University of applied sciences?

\subsection{Sub-questions}
\begin{enumerate}
\item How technical are the students and researchers from the Makerslab on the Amsterdam University of applied sciences?
\begin{enumerate}
\item What documentation is needed and how detailed should it be
\end{enumerate}
\item What improves the usability of the product when taking the structure plane of user experience into account?
\item How can the data be made available such that the students \& researchers from the Minor Makerslab can use it?
\end{enumerate}

\section{Review of literature}
This has no content yet and needs to be filled in. 

\section{Methods}
This has no content yet and needs to be filled in. 

\section{Results}
This has no content yet and needs to be filled in. 

\section{Discussion}
This has no content yet and needs to be filled in. 

\section{Limitations}
This has no content yet and needs to be filled in. 

\section{Conclusion}
This has no content yet and needs to be filled in. 

\bibliography{Research_Report_Monitoring_Box}

\appendices
\section{Baseline of questions asked during user tests}
\begin{enumerate}
\item Which field do you have expertise in?
\item What is your level of technology in hardware?
\item What is your level of technology in software
\item How does the explanation of the parts in the manual help you using the monitoring box?
\item Do you think developers with similar technical levels can make and use the monitoring box as well?\item What could be improved when looking at the monitoring box?
\item What could be improved when looking at the manual?
\end{enumerate}

\end{document}
